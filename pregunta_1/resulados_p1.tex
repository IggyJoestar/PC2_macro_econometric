\documentclass{article}
\usepackage{graphicx}
\usepackage{booktabs}
\usepackage{float}
\usepackage{array} % Para mejor control de columnas
\usepackage{siunitx} % Para alineación numérica

\begin{document}

\begin{table}[H]
\centering
\caption{Percentiles empíricos de $z_i$ vs  teóricos de la $\mathcal{N}(0,1)$\label{tab:percentiles}}
\begin{tabular}{l
            S[table-format=-1.3]
            S[table-format=-1.3]
            S[table-format=-1.3]
            S[table-format=-1.3]}
\toprule
\textbf{Percentil} & 
{\textbf{Teórico}} & 
{\textbf{Empírico}} & 
{\textbf{Empírico}} & 
{\textbf{Empírico}} \\
& {$\mathcal{N}(0,1)$} & {($T=5$)} & {($T=100$)} & {($T=1000$)} \\
\midrule
$P_{1\%}$     & -2.326 & -1.667 & -2.181 & -2.263 \\
$P_{2.5\%}$   & -1.960 & -1.536 & -1.842 & -1.899 \\
$P_{5\%}$     & -1.645 & -1.387 & -1.541 & -1.579 \\
$P_{10\%}$    & -1.282 & -1.173 & -1.249 & -1.273 \\
$P_{90\%}$    &  1.282 &  1.259 &  1.283 &  1.272 \\
$P_{95\%}$    &  1.645 &  1.814 &  1.697 &  1.687 \\
$P_{97.5\%}$  &  1.960 &  2.278 &  2.047 &  1.987 \\
$P_{99\%}$    &  2.326 &  2.796 &  2.516 &  2.393 \\
\bottomrule
\end{tabular}
\end{table}

\end{document}
